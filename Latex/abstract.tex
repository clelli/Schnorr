\newpage
\chapter*{Abstract}
\addcontentsline{toc}{chapter}{Abstract}
In 1991 on the \textit{Journal of Cryptology}, Claus Peter Schnorr published a paper titled \textit{"Efficient Signature Generation by Smart Cards"}, where he presented his idea for a new efficient signature scheme.\\
Even if it has so many interesting features and benefits, it has not been standardised yet. We present our implementation of \textit{Schnorr Signature} applied to Elliptic Curve Cryptography, which is based on the assumption of the Discrete Logarithm Problem.\\
Throughtout our dissertation, we explain our choices step by step and we illustrate the various properties of this signature, such as shortness, fast verification, but most of all \textit{additivity}. This one is not present in any other signature, and leads to a very important and innovative feature: \textit{multisignature}, a protocol through which a group of signers can generate a single joint signature on a common message.\\
We illustrate also our implementation of multisignature scheme and the several benefits brougth by this feature.\\
Finally, we introduce Elliptic Curve Digital Signature Algorithm, currently used in Bitcoin, showing how much improvements Schnorr Signature Algoritmh carries with it.